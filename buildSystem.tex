\documentclass[master.tex]{subfiles}
 
\begin{document}


Since the project is rather large (\textgreater 10000 Lines of Code spread over $\sim 100$ Files) a sophisticated build system is required and thus CMake has been choosen.
The components of the simulation are divided into CMake-Targets. There are multiple library targets which encapsulate derivatives, algorithms, helper methods and model equations as well as a few executable targets. One of these executable targets will be the desired simulation.\newline
The project folder structure looks like this:\newline

\begin{forest}
  for tree={
    font=\ttfamily,
    grow'=0,
    child anchor=west,
    parent anchor=south,
    anchor=west,
    calign=first,
    edge path={
      \noexpand\path [draw, \forestoption{edge}]
      (!u.south west) +(7.5pt,0) |- node[fill,inner sep=1.25pt] {} (.child anchor)\forestoption{edge label};
    },
    before typesetting nodes={
      if n=1
        {insert before={[,phantom]}}
        {}
    },
    fit=band,
    before computing xy={l=15pt},
  }
[t3g-cmake
 [extern \textit{Containing External Libraries}] 
 [src
   [algorithms
     [derivatives \textit{finite differences schemes etc.}]
     [poisson \textit{FFTW and SOR Solver}]
     [time \textit{Karniadakis Time Stepper}]
   ]
   [core \textit{Log, Matrix-Class, HDF5 Output}]
   [domain \textit{Shear Shifted Grid Domain}]
   [initialization]
   [models
     [isothermal \textit{Equations implementations}]
     [simulations \textit{Simulation Runner}]
   ]
   [performanceTests \textit{Small programs to evaluate various optimizations}]
   [plotting \textit{API for live plotting}]
   [cuda \textit{Cuda Implementations}]
 ]
 [t3g-config \textit{Default config files}]
 [plotter \textit{Small python program for data visualization}]
]
\end{forest}

Each folder that has C++ code has its own CMakeLists.txt where different build properties and the targets itself are defined (for further information about the CMake code refer to the documentation).\newline
The git project contains a \textit{Readme} file in which the build process is further described.

\end{document}