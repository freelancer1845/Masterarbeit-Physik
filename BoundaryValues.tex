To treat the domain boundaries ghost cells are implemented. Currently the simulation needs two ghost cells into every direction. The memory and performance impact is discussed in the \textit{Computer Science} section.
\subsection{X-Direction}
In X-Direction (i. e. polodial distance) periodic boundary conditions are implemented. This is achieved using the ghost cells. \todo{Describe External Forces}
\subsection{Y-Direction}
In Y-Direction (which is perpendicular to X but not in torodial direction) closed boundary conditions are implemented. One of the consequences is that there is an \textit{in and out flow} of vorticies simulating turbulence incoming from the not simulated domain. \todo{find reference for that}. The resolution in Y-Direction needs to be choose carefully such that vorticies that are mirrored to the other side wouldn't effect themselfes. 
\subsection{Z-Direction}
In Z-Direction there is a differentiation whether the X-coordinate lies in/on or outside the separatrix.
\paragraph{In/On Separatrix}
Here we assume closed field lines and thus employ closed boundary conditions.
\paragraph{Outside of Separatrix}
To simulate \todo{Ref-Ribero-Scott} divertors outside the separatrix so called \textit{limiters} are simulated by not connecting the $z=0 \And z=n_z$ planes but rather using open boundary conditions for the densities and potential, and Dirichlet boundary conditions for the velocities.  

\todo{Picture showing boundary conditions}