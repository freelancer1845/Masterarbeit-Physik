\documentclass[master.tex]{subfiles}
 
\newcommand{\meanxy}[1]{\left<#1\right>_{z,y}}
\begin{document}


\subsection{Connection to Navier-Stokes}
\label{sec:connection_navier_stokes}

The structure of the density and velocity equations closely resembles that of the incompressible Navier-Stokes equations.
There are some transformations necessary to see the connection and we restrict ourselves to the 2D case. We start with the incompressible Navier Stokes equations.

\begin{align}
    \partial_t v &= - \nabla p - v \cdot \nabla v + \Delta v
\end{align}

In the two-dimensional case this transformation
\begin{align}
    v_x &= \partial_y \psi \\
    v_y &= -\partial_x \psi \\
    \xi &= \Delta \psi
\end{align}
yields the following pressure-free formulation of the Navier-Stokes equations.

\begin{align}
    \partial_t \xi &= \left( \partial_x \psi \partial_y \xi - \partial_y \psi \partial_x \xi  \right) + \Delta \xi\\
    \partial_t \xi &= - \left[\xi, \psi\right] + \Delta \xi \label{eq:vorticity-streamline-potential} \\
\end{align}

In a $\delta$f framework and without thermal effects ($\tau_i = 0$) the evolution equations for the (gyro-averaged) densities (\autoref{eq:electron_density} and \autoref{eq:ion_density}) can be combined (considering the charge of the species $z_s$) and transformed into  \autoref{eq:vorticity-streamline-potential} by using a $\delta$f form of the polarization equation (\autoref{eq:polarization}).\newline
This only serves as motivation to use terminology from flow equations. We can now define turbulent and zonal flow in the following way:

\begin{align}
    \Gamma(z,x,y) &= \partial_y \phi_e :\, Turbulent\,Flow\\
    \Lambda(z,x,y) &= \partial_x \phi_e :\, Poloidal\,Flow
\end{align}

Here the turbulent flow gives us information on how much of our quantities is transported outwards into the \ac{SOL} by $\mathrm{E}\times\mathrm{B}$ convection and the poloidal flow represents perpendicular (to the magnetic field) transport parallel to the separatrix.

\subsection{Simulation Quantities}\label{sec:simulation_quantities}



In order to characterize the simulation results some quantities are defined. Often we are interested in the radial profile of our quantities thus we define the radial profile of a quantity $\mathcal{A}$, as:
\begin{equation}
    \left<\mathcal{A}\right>_y = \frac{1}{L_y} \int_{[y]} A(z,x,y) \, dy
\end{equation}
which essentially is the mean value over the y-direction,
and the zonal profile as:
\begin{equation}
    \left<\mathcal{A}\right>_{z, y} =\frac{1}{L_yL_z} \int_{[z, y]} A(z,x,y) \, dzdy
\end{equation}
Here $[z,x,y]$ should represent the full domain and $[z,y]$ the subdomain $[L_z \times L_y]$. Further on $S$ is defined as the separatrix position in $x$-direction to define two subsets of the full domain by $[z,y,x\leq S]$ as the core region and $[z,y,x > \mathcal{S}]$ as the \ac{SOL} region.



We now define these quantities for further evaluation in \autoref{sec:polarization_equation_evaluation}:

\begin{itemize}
    \item $\overline{\Gamma}_{Core} \propto  \left<\Gamma\right>_{z,y, x \leq \mathcal{S}}$ : \textit{Turbulent Flow in Core Plasma}
    \item  $\overline{\Gamma}_{Core} \propto  \left<\Gamma\right>_{z,y, x > \mathcal{S}}$ : \textit{Turbulent Flow in \ac{SOL} Plasma}
    \item $\Gamma_{Core}^*(x) \propto \left<\Tilde{n}_s \cdot \Gamma\right>_{z,y}$ : \textit{Turbulent Flux at $x$}
    \item $\meanxy{\phi_e}$ : \textit{Zonal Potential}
    \item $\partial_x \meanxy{\phi_e}$ : \textit{Zonal Flow}
    \item $\partial_{xx}\meanxy{\phi_e}$ : \textit{Zonal Vorticity}
    \item $E_n \propto \left< \tau_s \sum_s n_s^2 \right>_{z, x, y,}$ : \textit{Full Density weighted by temperature ratio}
    \item $\mathrm{E}_{v_\parallel} = \hat{\epsilon} \cdot \left<\sum_s \mu_s v_{s\parallel} \right>_{z, x, y}$ : \textit{Parallel Kinetic Energy}
\end{itemize}


\end{document}
