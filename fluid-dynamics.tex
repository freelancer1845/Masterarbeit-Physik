\documentclass[master.tex]{subfiles}
 
\newcommand{\meanxy}[1]{\left<#1\right>_{z,y}}
\begin{document}


\subsection{Connection to Navier-Stokes}
\label{sec:connection_navier_stokes}

The structure of the density and velocity equations closely resembles that of the incompressible Navier-Stokes equations.
There are some transformations necessary to see the connection and we restrict ourselves to the 2D case. We start with the incompressible Navier Stokes equations.

\begin{align}
    \partial_t v &= - \nabla p - v \cdot \nabla v + \Delta v
\end{align}

In the two-dimensional case this transformation
\begin{align}
    v_x &= \partial_y \psi \\
    v_y &= -\partial_x \psi \\
    \xi &= \Delta \psi
\end{align}
yields the following pressure-free formulation of the Navier-Stokes equations.

\begin{align}
    \partial_t \xi &= \left( \partial_x \psi \partial_y \xi - \partial_y \psi \partial_x \xi  \right) + \Delta \xi\\
    \partial_t \xi &= - \left[\xi, \psi\right] + \Delta \xi
\end{align}

Now we can compare this to the moment equations written down earlier and identify the streamline potential $\psi$ with the electric potential and the vorticity $\xi$ with density or velocity. This only serves as motivation to use terminology from flow equations. We can now define turbulent and zonal flow in the following way:

\begin{align}
    \Gamma(z,x,y) &= \partial_y \phi_e :\, Turbulent\,Flow\\
    \Lambda(z,x,y) &= \partial_x \phi_e :\, Poloidal\,Flow
\end{align}

Here the turbulent flow gives us information on how much of our quantities is transported outwards into the \ac{SOL} and the poloidal flow represents perpendicular (to the magnetic field) transport parallel to the separatrix. They are connected to the perpendicular velocities via the Lorentz-force since $E = -\nabla \phi$.

\subsection{Simulation Quantities}\label{sec:simulation_quantities}

In order to characterize the simulation results some quantities are defined. Often we are interested in the radial profile of our quantities thus we define the radial profile of a quantity $\mathcal{S}$, in discretized matrix form, as:
\begin{equation}
    (\left<\mathcal{S}\right>_y)_i = \frac{1}{n_y} \sum_{j=0}^{n_y} S_{i, j}
\end{equation}
which essentially is the mean value over the y-direction.\\
Further on we define these quantities for further evaluation in \autoref{sec:polarization_equation_evaluation}:

\begin{itemize}
    \item $\overline{\Gamma}_{Core} = \alpha \sum_{z, y, x \leq \mathcal{S}} \Gamma(z,x,y)$ : \textit{Turbulent Flow in Core Plasma}
    \item $\overline{\Gamma}_{SOL} = \alpha \sum_{z, y, x > \mathcal{S}} \Gamma(z,x,y)$ : \textit{Turbulent Flow in \ac{SOL} Plasma}
    \item $\meanxy{\phi_e}$ : \textit{Zonal Potential}
    \item $\partial_x \meanxy{\phi_e}$ : \textit{Zonal Flow}
    \item $\partial_{xx}\meanxy{\phi_e}$ : \textit{Zonal Flow Potential / Vorticity}
    \item $E_n \propto \int_{[z,x,y]} \cdot \tau_s \sum_s n_s^2$ : \textit{Full Density}
    \item $\mathrm{E}_{v_\parallel} = \hat{\epsilon} \cdot \int_{[z,x,y]} \sum_s \mu_s v_{s\parallel} \, dxdydz$ : \textit{Parallel Kinetic Energy}
\end{itemize}

Here the $\int_{[z,x,y]}$ essentially means the summation of each grid point divided by the total grid size ($n_z \cdot n_x \cdot n_y$).

\end{document}
