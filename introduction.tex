\documentclass[master.tex]{subfiles}
 
\begin{document}

Turbulence is still one of the greatest problems for physics. The non linear nature of its description makes it almost always impossible to solve a given set of equations that \textit{contains} turbulence analytically. Therefore extensive computational efforts are being made to tackle such systems of equations numerically \cite{IntroTurbulence}.\newline
In todays fusion reactor experiments it is thought that turbulence is the underlying principle for various instabilities and transport effects that need to be understood to predict experimental results. A special focus lies on the boundary of the plasma where it \textit{touches} the wall of the reactor. At this \textit{edge region} high density and temperature gradients are observed leading to different (turbulent) phenomena that may destabilize the plasma and/or damage the reactor.\newline
Because of the long range interaction properties of a plasma and the high mass difference between ions and electrons many different length and time scales are involved in this region.\newline
The situation can be modeled using maxwells equations. A popular approach to this is the gyro kinetic description of the system which reduces the 6-dimensional description of a particle $(\vec{x},\vec{v})$ into a 5-dimensional description which is valid since in our situation a strong magnetic field is present. To solve this numerically a so called particle in a cell approach is taken which is computational extremely heavy.\newline
Another approach is a statistical one that simplifies the gyro-kinetic equations and treats the ensemble of particles in the plasma as a fluid resulting in the gyro-fluid equations further reducing computational complexity. Characteristically for both models is the \textit{polarization equation} which has the form $\nabla N \nabla \phi = F$ and describes the electric potential which drives most of the dynamics. Since this non-linear poisson equation is not easily solved a typically linearization is to assume that $\nabla N \approx \tilde{N} \nabla$ which is valid for small density fluctuations in respect to the background density (\textit{delta-f vs full-f}). Since the computational effort needed to evaluate the gyro-fluid model is much less than the gyro-kinetic model it can be used to investigate this approximation and its effect on the simulation without the complexity of a simulation code that has to be run on a cluster.\newline
In this master thesis the Full-F Gyro Fluid model equations are presented and the numerical methods used to solve this system of equations are introduced. Further on some implementation details are presented as well as some performance improvements are employed. Later on these performance improvements are evaluated and different linearizations for the polarization equation are compared to the non linearized case.\newline
In between there is also some information on how the model is translated into actual code and how the simulation is structured.


\end{document}