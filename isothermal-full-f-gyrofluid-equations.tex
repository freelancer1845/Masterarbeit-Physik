\documentclass[master.tex]{subfiles}
 
\begin{document}

The equations are stated as they are. A mathematical derivation can be found in \cite{HeldDisseration}. It is essentially a set of six equations. Two density equations (\autoref{eq:electron_density}, \autoref{eq:ion_density}), two parallel velocity equations (\autoref{eq:electron_velocity}, \autoref{eq:ion_velocity}), the polarization equation (\autoref{eq:polarization}) and a closure approximation for the gyro averaging operator $\Gamma_1$.

\subsection{Density Equations}

\begin{align}
    \overbrace{\partial_t n_e}^{Variation} &=
    \overbrace{-\delta^{-1} \left[\Tilde{n}_e, \phi \right]_{\perp}}^{\perp \, Convection}
    \overbrace{
    - \mathcal{K}(\Tilde{n}_e)
    - \mathcal{K}(\phi)
    }^{\perp \, Diffusion}
    \overbrace{
    - \frac{1}{B(z)}\nabla_\parallel (v_{e\parallel} n_e)
    }^{\parallel \, Diffusion}
    \overbrace{
    + \mu_\parallel \nabla_\parallel^2\Tilde{n}_e
    - \mu_\perp \Delta^2_\perp\Tilde{n}_e
    }^{Sources, Sinks}
    \label{eq:electron_density}
    \\
    \partial_t N_i &=
    -\delta^{-1} \left[\Tilde{N}_i, \psi \right]_{\perp}
    - \mathcal{K}(\Tilde{N}_i)
    - \mathcal{K}(\Tilde{\psi_i})
    - \frac{1}{B(z)}\nabla_\parallel (v_{i\parallel} N_i)
    + \mu_\parallel \nabla_\parallel^2\Tilde{N}_i
    - \mu_\perp \Delta^2_\perp\Tilde{N}_i \label{eq:ion_density}
\end{align}

where

\begin{align}
    \mathcal{K}(\lambda) &= (\kappa_x \partial_x + \kappa_y \partial_y)\,\lambda\\
    \kappa_x &= mcv \cdot sin(\theta)\\
    \kappa_y &= mcv \cdot cos(\theta) + shear\\
    \left[A, B\right]_\perp &= \partial_x A \cdot \partial_y B - \partial_x B \cdot \partial_y A\\
    \nabla_\parallel \lambda &= \Tilde{\partial_{z}} \lambda
\end{align}
\begin{itemize}
    \item $\Tilde{\partial_{z}}$ represents the shear-shifted derivative in z-direction.
    \item $\Tilde{X} := log(X)$
    \item $B(z) := $ Magnetic-Flux in Z-Plane
\end{itemize}


\autoref{eq:electron_density} represents the evolution of the electron density field.
\autoref{eq:ion_density} represents the evolution of the density of gyro-averaged species (i. e. ions). Since the electrons gyro-radius is small compared to the ion gyro-radius, the effects of electron gyration can be neglected.\todo{Find a reference or write equations down} \newline
The last two terms in both equations are needed for numerical stability of our schemes but may be regarded as sources and sinks. 


\subsection{Parallel Velocity Equations}
The velocity equations are restricted to the parallel direction which follows from the "frozen-flux-theorem" which states that particle motion is strongly restricted to the magnetic field lines \todo{Ref}. Thus it is convenient to align the parallel direction ($z$) to the magnetic flux.


\begin{align}
\begin{split}
    \alpha_e &=  -\delta^-1 \left[v_{e\parallel}, \phi_\perp\right]
    - 2\tau_e \mathcal{K}( v_{e\parallel})
    - 2\tau_e^2 \mathcal{K}(\Tilde{n}_e)
    - \tau_e \mathcal{K}( \phi)
    - \nabla_\parallel (\phi + \tau_e n_e)
    - \hat{c}\, \mathrm{J}\\
    & \quad + \mu_\parallel \nabla_\parallel^2\Tilde{\alpha}_e 
    - \mu_\perp \Delta^2\Tilde{\alpha}_e \label{eq:electron_velocity}
\end{split}\\
\begin{split}
    \alpha_i &= -\delta^-1  \left[v_{i\parallel}, \phi_\perp\right]
    - 2\tau_i \mathcal{K}(v_{i\parallel})
    - 2\tau_i^2 \mathcal{K}(N_i)
    - \tau_i \mathcal{K}(\psi)
    - \nabla_\parallel (\psi + \tau_i N_i)
    - \hat{c}\, \mathrm{J}\\
    &\quad + \mu_\parallel\nabla_\parallel^2\Tilde{\alpha}_i 
    - \mu_\perp \Delta^2\Tilde{\alpha}_i \label{eq:ion_velocity}
\end{split}
\end{align}

where ($s$ denotes electron and ion species)

\begin{itemize}
    \item $\alpha_s := \mu_s \hat{\epsilon} \cdot v_{s\parallel} + \hat{\beta} \cdot A_{\parallel}$
    \item $\mathrm{J} := \sum_s z_s \cdot n_s \cdot v_{s\parallel}$
    \item $z_s :=$ charge of species particles (i. e. -1 for electrons)
\end{itemize}

As a simplification we assume $\hat{\beta} \approx 0$ and thus Amperes Law is not stated here. Nonetheless to closely represent the equations implemented in the simulation code the $\alpha$-notation is used here.

\subsection{Polarization Equation}

\begin{align}
    \nabla \cdot \left( \sum_s \mu_s n_s \right) \nabla \phi &= z_e n_e + \sum_i z_i \, \cdot \, N_i \label{eq:polarization}
\end{align}
The polarization equation connects the individual species. On the right hand side essentially stands the (gyro-averaged) charge distribution. The left-hand side contains \ac{FLR} effects via the non-linearity. A typical assumption is that $\alpha := \sum_s \mu_s n_s$ can be expanded in space and or time ($\alpha \approx \alpha_0 + \alpha_1 \epsilon + ...$ where $\epsilon$ represents the Larmor radius length scale) and that $\alpha_1 \epsilon \ll \alpha_0$ as well as $\alpha_0$ is constant. Three different simplifications of that kind will be presented in \autoref{sec:polarization-linearizations} and evaluated in \autoref{sec:evaluation}. 


\subsection{Gyro-Averaging Operator}
Since we have a Gryofluid model we need a transformation from the actual quantities into the gyro averaged quantities. In our case we only need $\Gamma_1$ and $\Gamma_1^\dagger$. A derivation and further explanation can be found in \cite{HeldDisseration}.
\begin{align}
    \Gamma_1 = \left(1- \frac{1}{2} \rho^2 \nabla_\perp^2\right)^{-1}
\end{align}


\section{Approximations of the Polarization Equation} \label{sec:polarization-linearizations}
\autoref{eq:polarization} is a general form of the more familiar Poisson equation
\begin{equation}\label{eq:linear-poisson-equation}
    \Delta \Psi = \rho 
\end{equation}
whereas \autoref{eq:polarization} has the following form
\begin{equation}\label{eq:general-poisson}
    \nabla N(x,y) \nabla \Psi = \rho(x,y).
\end{equation}

Solving \autoref{eq:linear-poisson-equation} is much simpler and thus it is desirable to bring \autoref{eq:general-poisson} into the same form. Presented are three different linearizations. These are evaluated and compared later.

\subsection{Constant Background Linearization}

In \autoref{eq:general-poisson} we assume $N(x,y, t) \approx \overline{N}_0$ for all times where $\overline{N}_0=\frac{1}{L_yL_x}\int_{[x,y]} N(x,y, t = 0) dx dy$. \autoref{eq:general-poisson} reduces to:
\begin{equation}
    \Delta \Psi(x,y,t) = \frac{\rho(x,y,t)}{\overline{N}_0}
\end{equation}
This approximation would be valid if $N(x,y, 0) \approx N(x,y,t)$ and $\nabla N(x,y,0) \approx 0$.

\subsection{Time-independent Radial Background Profile}
In \autoref{eq:general-poisson} we assume $N(x,y,t) \approx \left<N(x,t = 0)\right>_y$. This reduces the general poisson equation to:
\begin{equation}
    \partial_x \left<N(x,t = 0)\right>_y \partial_x \Psi(x,y,t) + \left<N(x,t = 0)\right>_y \partial_{yy} \Psi(x,y,t) = F(x,y,t);
\end{equation}
This would be valid if $\partial_y N \approx 0$ and $N(x,y,t) \approx N(x,y, t= 0)$.

\subsection{Time-dependent Radial Background Profile}
In \autoref{eq:general-poisson} we assume $N(x,y,t) \approx \left<N(x,t)\right>_y$. This reduces the general poisson equation to:
\begin{equation}
    \partial_x \left<N(x,t)\right>_y \partial_x \Psi(x,y,t) + \left<N(x,t)\right>_y \partial_{yy} \Psi(x,y,t) = F(x,y,t);
\end{equation}
This would be valid if $\partial_y N \approx 0$.
\newline
Evaluation and results are found in \autoref{sec:polarization_equation_evaluation}.

\end{document}