The equations are stated as they are. A mathematical derivation can be found in \cite{HeldDisseration}. It is essentially a set of six equations. Two density equations (\autoref{eq:electron_density}, \autoref{eq:ion_density}), two velocity equations (\autoref{eq:electron_velocity}, \autoref{eq:ion_velocity}), the polarization equation (\autoref{eq:polarization}) and a closure approximation for the gyro averaging operator $\Gamma_1$.

\subsection{Density Equations}

\begin{align}
    \partial_t n_e &=
    -\delta^{-1} \left[\Tilde{n}_e, \phi \right]_{\perp}
    - \mathcal{K}(\Tilde{n}_e)
    - \mathcal{K}(\phi)
    - \frac{1}{B(z)}\nabla_\parallel (v_{e\parallel} n_e)
    + \mu_\parallel \nabla_\parallel^2\Tilde{n}_e
    - \mu_\perp \Delta^2_\perp\Tilde{n}_e \label{eq:electron_density}
    \\
    \partial_t N_i &=
    -\delta^{-1} \left[\Tilde{N}_i, \psi \right]_{\perp}
    - \mathcal{K}(\Tilde{N}_i)
    - \mathcal{K}(\Tilde{\psi_i})
    - \frac{1}{B(z)}\nabla_\parallel (v_{i\parallel} N_i)
    + \mu_\parallel \nabla_\parallel^2\Tilde{N}_i
    - \mu_\perp \Delta^2_\perp\Tilde{N}_i \label{eq:ion_density}
\end{align}

where

\begin{align}
    \mathcal{K}(\lambda) &= (\kappa_x \partial_x + \kappa_y \partial_y)\,\lambda\\
    \kappa_x &= mcv \cdot sin(\theta)\\
    \kappa_y &= mcv \cdot cos(\theta) + shear\\
    \left[A, B\right]_\perp &= \partial_x A \cdot \partial_y B - \partial_x B \cdot \partial_y A\\
    \nabla_\parallel \lambda &= \Tilde{\partial_{z}} \lambda
\end{align}
\begin{itemize}
    \item $\Tilde{\partial_{z}}$ represents the shear-shifted derivative in z-direction.
    \item $\Tilde{X} := log(X)$
    \item $B(z) := $ Magnetic-Flux in Z-Plane
\end{itemize}


\newline
\autoref{eq:electron_density} represents the evolution of the electron density field.
\autoref{eq:ion_density} represents the evolution of the density of gyro-averaged species (i. e. ions). Since the electrons gyro-radius is small compared to the ion gyro-radius, the effects of electron gyration can be neglected.\todo{Find a reference or write equations down} \newline
The last two terms in both equations are needed for numerical stability of our schemes but may be regarded as sources and sinks. 


\subsection{Parallel Velocity Equations}
The velocity equations are restricted to the parallel direction which follows from the "frozen-flux-theorem" which states that particles motion is strongly restricted to the magnetic field lines \todo{Ref}. Thus it is convenient to align the parallel direction ($z$) to the magnetic flux.


\begin{align}
    \alpha_e &=  -\delta^-1 \left[v_{e\parallel}, \phi_\perp\right]
    - 2\tau_e \mathcal{K}( v_{e\parallel})
    - 2\tau_e^2 \mathcal{K}(\Tilde{n}_e)
    - \tau_e \mathcal{K}( \phi)
    - \nabla_\parallel (\phi + \tau_e n_e)
    - \hat{c}\, \mathrm{J}
    + \mu_\parallel \nabla_\parallel^2\Tilde{\alpha}_e 
    - \mu_\perp \Delta^2\Tilde{\alpha}_e \label{eq:electron_velocity} \\
    \alpha_i &= -\delta^-1  \left[v_{i\parallel}, \phi_\perp\right]
    - 2\tau_i \mathcal{K}(v_{i\parallel})
    - 2\tau_i^2 \mathcal{K}(N_i)
    - \tau_i \mathcal{K}(\psi)
    - \nabla_\parallel (\psi + \tau_i N_i)
    - \hat{c}\, \mathrm{J}
    + \mu_\parallel\nabla_\parallel^2\Tilde{\alpha}_i 
    - \mu_\perp \Delta^2\Tilde{\alpha}_i \label{eq:ion_velocity}
\end{align}

where ($s$ denotes electron and ion species)

\begin{itemize}
    \item $\alpha_s := \mu_s \hat{\epsilon} \cdot v_{s\parallel} + \hat{\beta} \cdot A_{\parallel}$
    \item $\mathrm{J} := \sum_s z_s \cdot n_s \cdot v_{s\parallel}$
    \item $z_s :=$ charge of species particles (i. e. -1 for electrons)
\end{itemize}

As a simplification we assume $\hat{\beta} \approx 0$ and thus Amperes Law is not stated here. Nonetheless to closely represent the equations implemented in the simulation code the $\alpha$-notation is used here.

\subsection{Polarization Equation}

\begin{align}
    \nabla \cdot \left( \sum_s \mu_s n_s \right) \nabla \phi &= z_e n_e + \sum_i z_i \, \cdot \, N_i \label{eq:polarization}
\end{align}
The polarization equation connects the individual species. On the right hand side essentially stands the (gyro-averaged) charge distribution. \todo{Write done which section} The left-hand side contains finite-lamor-radius (FLR) effects via the non-linearity which will be elaborated on in.

\subsection{Gyro-Averaging Operator}
\begin{align}
    \Gamma_1 = \left(1- \frac{1}{2} \rho^2 \nabla_\perp^2\right)^{-1}
\end{align}


\subsection{Connection to Navier-Stokes}
\label{sec:connection_navier_stokes}

The structure of the density and velocity equations closely resembles that of the incompressible navier-stokes equations. The form given here is taken from \cite{KARNIADAKIS1991414} equation \textit{1.a}.

\begin{align}
    \partial_t \Psi &= -\nabla p + \mu \mathbf{L}(\Psi) + \mathbf{N}(\Psi)\\
    \mathbf{L}(\Psi):&= \nabla^2\Psi = \nabla(\nabla\cdot\Psi)-\nabla \times (\nabla \times \Psi)\\
    \mathbf{N}(\Psi):&= - \frac{1}{2}\left[\Psi \cdot \nabla \Psi + \nabla(\Psi \cdot \Psi)\right]
\end{align}

There are some transformations necessary to see the connection. We start with the incompressible navier stokes equations.

\begin{align}
    \partial_t v &= - \nabla p - v \cdot \nabla v + \Delta v
\end{align}

In the two-dimensional case this transformation
\begin{align}
    v_x &= \partial_y \psi \\
    v_y &= -\partial_x \psi \\
    \phi &= \Delta \psi
\end{align}
results in the following pressure-free formulation of the navier-stokes equations.

\begin{align}
    \partial_t \phi &= \left( \partial_x \psi \partial_y \phi - \partial_y \psi \partial_x \phi  \right) + \Delta \phi\\
    \partial_t \phi &= - \left[\phi, \psi\right] + \Delta \phi
\end{align}

Now we can compare this to the moment equations written down early and identify the streamline potential $\psi$ with the electric potential and $\phi$ with density or velocity. This only serves as motivation to use terminology from flow equations. We can now define turbulent and zonal flow in the following way:

\begin{align}
    \Gamma(z) &= \partial_y \phi_e :\, Turbulent\,Flow\\
    \Lambda(z) &= \partial_x \phi_e :\, Zonal\,Flow
\end{align}

Here the turbulent flow gives us information on how much of your quantities is transported outwards into the SOL and the zonal flow represents transport parallel to the separatrix.